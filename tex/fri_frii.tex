\documentclass{article}

\usepackage{amssymb}
\usepackage[margin=1cm]{geometry}
\usepackage{tikz}
\usetikzlibrary{arrows,fit,positioning,shapes,calc}
\usepackage{hyperref}

\title{Design document for FR-I and FR-II classification demo}
\author{Matthew Alger, Julie Banfield and Cheng Soon Ong}
\date{19 April 2018} % without \date command, current date is supplied



%%%%%%%%%%%%%%
% Tikz style
%%%%%%%%%%%%%%

%-----#1 name of the node, #2 relative location, #3 other node, #4 label
\def\mlsystem[#1,#2,#3]#4;{
  \node[draw, align=left, name=#1, #2 = of #3, color=blue,fill=white,rounded corners=3] {#4};
}

%-----#1 name of the node, #2 relative location, #3 other node, #4 label
\def\lib[#1,#2,#3]#4;{
  \node[draw, name=#1, #2 = of #3, color=red,fill=white,rounded corners=3] {#4};
}

%-----#1 name of the node, #2 relative location, #3 other node, #4 label
\def\web[#1,#2,#3]#4;{
  \node[draw, name=#1, #2 = of #3, color=black,fill=blue!30] {#4};
}

%-----#1 name of the node, #2 relative location, #3 other node, #4 label
\def\csv[#1,#2,#3]#4;{
  \node[draw, name=#1, #2 = of #3, color=black,fill=gray,minimum height=50] {#4};
}


%-----database
%-----#1 height, #2 width, #3 aspect, #4 name of the node,
%-----#5 coordinate, #6 label
\def\database[#1,#2,#3,#4,#5]#6{%
  \node[draw, cylinder, alias=cyl, shape border rotate=90, aspect=#3, %
  minimum height=#1, minimum width=#2, outer sep=-0.5\pgflinewidth, %
  color=orange!40!black, left color=orange!70, right color=orange!80, middle
  color=white] (#4) at #5 {};%
  \node at #5 {#6};%
  \fill [orange!30] let \p1 = ($(cyl.before top)!0.5!(cyl.after top)$), \p2 =
  (cyl.top), \p3 = (cyl.before top), \n1={veclen(\x3-\x1,\y3-\y1)},
  \n2={veclen(\x2-\x1,\y2-\y1)} in (\p1) ellipse (\n1 and \n2); }

%------dataflow
%------#1 source, #2 target, #3 bend left or right
%------#4 label
\def\dataflow[#1,#2,#3]#4{%
  \path (#1) edge[->, thick, shorten <=2pt, shorten >=2pt, bend #3] node [left] {#4} (#2);
}

%------todo
%------#1 source, #2 target, #3 bend left or right
%------#4 label
\def\todoflow[#1,#2,#3]#4{%
  \path (#1) edge[color=red, ->, thick, shorten <=2pt, shorten >=2pt, %
  bend #3] node [right] {#4} (#2);
}


\begin{document}

\maketitle

\begin{center}
  \begin{tikzpicture}
    \database[60,50,1.6,db,(0,0)] {\begin{tabular}{c}FIRST\\TGSS\end{tabular}};
    \mlsystem[pred,below,db] {\begin{tabular}{c}
      {\bf Predictor}\\$f_I:\quad$component$\to[0,1]$\\$f_{II}:\quad$component$\to[0,1]$\\
      component = $\mathbb{R}^d\in$Features
    \end{tabular}};
    \mlsystem[rec,left,db] {\begin{tabular}{c}
      {\bf Recommender}\\$r:\quad$component$\to\mathbb{R}$\\
      component = $[0,1]^2$ prediction
    \end{tabular}};
    \lib[fetch,above,db] {\begin{tabular}{c}
      {\bf Image retriever}\\$g:\quad$component$\to$subject\\
      component = FIRST index
    \end{tabular}};
    \csv[features,right,db] {Features};
    \web[zoo,above right,features] {\begin{tabular}{c}
      {\bf Label with Zooniverse}\\
      subjects$\to$classifications
    \end{tabular}};

    \todoflow[db,fetch,left] {};
    \todoflow[rec,fetch,left] {};
    \todoflow[fetch,zoo,left] {};
    \todoflow[zoo,features,left] {};
    \todoflow[zoo,pred,left] {Callback or trigger};
    \todoflow[db,fetch,left] {};
    \dataflow[db,features,left] {};
    \dataflow[pred,rec,left] {};
    \dataflow[features,pred,left] {};
  \end{tikzpicture}
\end{center}

Notes about the figure:
\begin{itemize}
  \item Suggest using a numerical index, which is implicit in the diagram above.
  \item ``component'' has different meanings in different scripts
  \item ``subject'' is the zooniverse subject
\end{itemize}

Workflow, beginning at the database (cylinder in the middle):
\begin{enumerate}
  \item Construct a set of features using\\ \url{https://github.com/jbanfield/broome/blob/master/nearest_neighbour.ipynb}
  \item Seed with a small set of labels
  \item Construct two classifiers using \texttt{acton}:
    \begin{itemize}
      \item FR-I vs all other objects
      \item FR-II vs all other objects
    \end{itemize}
  \item Apply both classifiers to all unlabelled features
  \item Use the recommender from \texttt{acton} to construct a score,
    and choose a small number of components (say 10) to get labelled.
  \item Get images from FIRST and TGSS of the small number of components,
    and upload to zooniverse.
  \item Querying FIRST images and catalogue.\\ \url{https://github.com/MatthewJA/ask-first/blob/master/ask_first/__init__.py}\\
  \url{https://github.com/MatthewJA/second/blob/master/get_data.py}
  \item Once labelled, add labels to the feature file
  \item Trigger a re-training of the predictors and the whole cycle.
\end{enumerate}


\end{document}
